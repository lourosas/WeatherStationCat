\documentclass[letterpaper]{article}
\usepackage{latexsym}
\usepackage{amsbsy}
\usepackage{graphicx}
\begin{document}
\noindent
\textbf{Use Case U1:  View the Latest Measured Weather Data}\\
\textbf{Scope:  }Weather Station\\
\textbf{Level:  }User Goal\\
\textbf{Primary Actor:  }User\\
\textbf{Scenario:  }View Dew Point Data\\
\textbf{Related Use Cases:  }U2: \textit{The User shall have the
ability to select different types of Views in viewing the Measured
Weather Data}\\ U4:  \textit{The User shall have the ability to
choose the Units to view the Current Weather Data}\\
\textbf{Stakeholders \& Interests:  }
\begin{itemize}
\item User:  Who wants to view the Dew Point Data
\end{itemize}
\textbf{Preconditions:  }The Thermometer is online and measuring.\\
The Hygrometer is online and Measuring.\\
\textbf{Success Gurantees:  }The User views the current Dew Point
Data.\\\\
\textbf{Main Success Scenario:  }\\
\begin{tabular}{|p{6cm}|p{6cm}|}\hline
\textbf{User} & \textbf{System}\\\hline
1.  Request to view the Dew Point Data & \\\hline
& 2.  Displays the Dew Point Data in the currently requested
units\\\hline
& 3.  Displays the Time (Day, Date, Time, Time Zone) of the
Dew Point Calculation\\\hline
\end{tabular}\\\\
\textbf{Extensions:  }\\
1a.  If the view of the System is already displaying the Dew Point
data, then the System will take no action.\\
1b.  If the System is not running, then User will start the System
first to view the Dew Point Data.\\
2a,3a.  If the Temperatur Sensor is not working, then the System
alerts the User of the Issue.\\
2b,3b.  If the Hygrometer is not working, then the System alerts the
User of the Issue.\\
\textbf{Frequecy Of Occurence: } Depends upon the Measurement
Frequency set in the System.\\\\
\textbf{Open Issues: } None
\end{document}
